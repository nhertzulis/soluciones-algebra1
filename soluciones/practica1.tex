\section*{Práctica 1 - Conjuntos, Relaciones y Funciones}

\subsection*{Conjuntos}

\begin{enumerate}
\item % Ejercicio 1
  \begin{multicols}{2}
  \begin{enumerate}
  \item $1 \in A$ Verdadero.
  \item $\{1\} \subseteq A$ Verdadero.
  \item $\{2, 1\} \subseteq A$ Verdadero.
  \item $\{1, 3\} \in A$ Falso.
  \item $\{2\} \in A$ Falso.
  \end{enumerate}
  \end{multicols}
\item % Ejercicio 2
  \begin{multicols}{2}
  \begin{enumerate}
  \item $3 \in A$ Falso.
  \item $\{3\} \subseteq A$ Falso.
  \item $\{3\} \in A$ Verdadero.
  \item $\{\{3\}\} \subseteq A$ Verdadero.
  \item $\{1, 2\} \in A$ Verdadero.
  \item $\{1, 2\} \subseteq A$ Falso.
  \item $\{\{1, 2\}\} \subseteq A$ Verdadero.
  \item $\{\{1, 2\}, 3\}\subseteq A$ Falso.
  \item $\emptyset \in A$ Falso.
  \item $\emptyset \subseteq A$ Verdadero.
  \item $A \in A$ Falso.
  \item $A \subseteq A$ Verdadero.
  \end{enumerate}
  \end{multicols}
\item % Ejercicio 3
  \begin{multicols}{2}
  \begin{enumerate}
  \item $A \subseteq B$ pues $1 \in B$, $2 \in B$ y $3 \in B$.
  \item $A \not\subseteq B$ pues $3 \in A$ pero $3 \notin B$.
  \item $A \not\subseteq B$ pues $\frac{5}{2} \in A$ pero $\frac{5}{2} \notin B$.
  \item $A \not\subseteq B$ pues $\emptyset \in A$ pero $\emptyset \notin B$.
  \end{enumerate}
  \end{multicols}
\item % Ejercicio 4
  \begin{multicols}{2}
  \begin{itemize}
  \item $A \cap B = \{3, 7, 11\}$
  \item $A \cup B = \{-1,1,3,-5,5,7,-8,8,11\}$
  \item $B - A = \{-1,-5,-8\}$
  \item $A \bigtriangleup B = \{-1,1,-5,5,-8,8\}$
  \end{itemize}
  \end{multicols}
\item % Ejercicio 5
  \begin{enumerate}
  \item $B \cap C = \emptyset$\newline Por lo tanto, $B \bigtriangleup C = B \cup C = \{1,\{3\},10,-2,\{1,2,3\},3\}$\newline Luego, $A \cap (B \bigtriangleup C)$ es igual a\newline \indent$\{1,-2,7,3\} \cap \{1,\{3\},10,-2,\{1,2,3\},3\} = \{1,-2,3\}$

  \item $A \cap B = \{1\}$\newline $A \cap C = \{-2,3\}$\newline $\therefore (A \cap B) \bigtriangleup (A \cap C) = \{1\} \bigtriangleup \{-2,3\} = \{1,-2,3\}$

  \item $A^c = V - A = \{\{3\},10,\{1,2,3\}\}$\newline
  $B^c = \{-2,7,\{1,2,3\},3\}$\newline
  $C^c = \{1,\{3\},7,10\}$\newline
  $\therefore A^c \cap B^c \cap C^c = \emptyset$
  \end{enumerate}
\item % Ejercicio 6
  Por la Ley de De Morgan:
  \begin{multicols}{2}
  \begin{itemize}
  \item $(A \cup B \cup C)^c = A^c \cap B^c \cap C^c$
  \item $(A \cap B \cap C)^c = A^c \cup B^c \cup C^c$
  \end{itemize}
  \end{multicols}
\item % Ejercicio 7
  Acá debería mostrar diagramas de Venn. Cuando tenga tiempo voy a buscar algún paquete de LaTeX para hacerlos.
\item % Ejercicio 8
  La idea de este ejercicio es buscar formas de expresar la resta de conjuntos y la diferencia simétrica utilizando uniones, intersecciones y complementos.
  \begin{multicols}{2}
  \begin{enumerate}
  \item $(A \cap B^c) \cup (B \cap C \cap A^c)$
  \item $(A \cup C) \cap (A \cap C)^c \cap B^c$
  \item $((A \cap B) \cup (B \cap C) \cup (C \cap A)) \cap (A \cap B \cap C)^c$
  \end{enumerate}
  \end{multicols}
\item % Ejercicio 9
  \begin{enumerate}
  \item $\{\emptyset,\{1\}\}$
  \item $\{\emptyset,\{a\},\{b\},\{a,b\}\}$
  \item $\{\emptyset,\{1\},\{\{1,2\}\},\{1,\{1,2\}\}\}$
  \item $\{\emptyset,\{a\},\{b\},\{c\},\{a,b\},\{a,c\},\{b,c\},\{a,b,c\}\}$
  \item $\{\emptyset,\{1\},\{a\},\{\{-1\}\},\{1,a\},\{1,\{-1\}\},\{a,\{-1\}\},\{1,a,\{-1\}\}\}$
  \item $\{\emptyset\}$
  \end{enumerate}
\item % Ejercicio 10
  Para probar que $P(A) \subseteq P(B) \Leftrightarrow A \subseteq B$ hay que probar que $P(A) \subseteq P(B) \Rightarrow A \subseteq B$ y que $P(A) \subseteq P(B) \Leftarrow A \subseteq B$. Es decir, hay que probar la implicación para los dos lados.
  \begin{itemize}
  \item $P(A) \subseteq P(B) \Rightarrow A \subseteq B$:\newline
  Demostración por el absurdo. Supongamos que $A \not\subseteq B$. Entonces existe $a \in A$ que no pertenece a B. Esto es equivalente a decir que $\{a\} \subseteq A$ pero $\{a\} \not\subseteq B$ (por definición de conjunto de partes). ¡Absurdo! Pues esta afirmación contradice a $P(A) \subseteq P(B)$. Luego, $P(A) \subseteq P(B) \Rightarrow A \subseteq B$.

  \item $A \subseteq B \Rightarrow P(A) \subseteq P(B)$:\newline
  Demostración por el absurdo. Supongamos que $P(A) \not\subseteq P(B)$. Entonces existe $a \in P(A)$ que no pertenece a $P(B)$. Esto es equivalente a decir que $a \subseteq A$ pero $a \not\subseteq B$ (por definición de conjunto de partes). ¡Absurdo! Pues esta afirmación contradice a $P(A) \not\subseteq P(B)$.
  \end{itemize}
  Acabamos de probar que $P(A) \subseteq P(B) \Leftrightarrow A \subseteq B$.
\item % Ejercicio 11
  Pendiente de completar.
\item % Ejercicio 12
  Sean los conjuntos\newline
  $A = \{\textnormal{``Argentinos"}\}$\newline
  $E = \{\textnormal{``Estudiantes de matemática de la facultad"}\}$\newline
  $M = \{\textnormal{``Materos, personas que toman mate"}\}$\newline
  Sabemos que
  \begin{itemize}
  \item $E \not\subseteq A$ (hay estudiantes extranjeros).
  \item $(M-A) \cap E = \emptyset$ (no hay materos extranjeros en la facultad).
  \end{itemize}
  ¿Estas premisas implican que $E \not\subseteq M$ (hay estudiantes que no toman mate)?\newline
  Sí, los estudiantes extranjeros no toman mate.\newline
  $E \not\subseteq A$ implica que existe una persona $p$ tal que $p \in E$ y $p \notin A$. Luego, $p \notin (M-A)$ pues $(M-A) \cap E = \emptyset$. Sabemos que $p \notin (M-A)$ y $p \notin A$. Entonces, $p \notin M$. Es decir, $p \in E$ pero $p \notin M$. Esto dice que $E \not\subseteq M$.
\end{enumerate}

\subsection*{Relaciones}

\begin{enumerate}
\setcounter{enumi}{17}

\item % Ejercicio 18 
  Dados dos conjuntos $A$ y $B$, un conjunto $R$ es una relación de $A$ en $B$ sii $R \in P(A \times B)$ o, equivalentemente, $R \subseteq A \times B$.
  \begin{multicols}{3}
  \begin{enumerate}
  \item Sí.
  \item No, pues $(3,2) \notin A \times B$.
  \item Sí.
  \item Sí.
  \item Sí.
  \item Sí.
  \item Sí.
  \item Sí.
  \end{enumerate}
  \end{multicols}
\item % Ejercicio 19
  Pendiente de completar.
\item % Ejercicio 20
  Dado un conjunto $A$, una relación $R$ de $A$ en $A$ (una relación en $A$) es
  \begin{itemize}
  \item Reflexiva sii $(\forall a \in A) a R a$.
  \item Simétrica sii $(\forall a,b \in A) (a R b \Rightarrow b R a)$.
  \item Transitiva sii $(\forall a,b,c \in A) ((a R b \wedge b R c) \Rightarrow a R c)$.
  \item Antisimétrica sii $(\forall a,b \in A) ((aRb \wedge bRa) \Rightarrow a = b)$\newline
  o, equivalentemente, $(\forall a,b \in A, a \neq b) ((a,b) \in R \Rightarrow (b,a) \notin R)$.
  \end{itemize}
  \begin{enumerate}
  \item
    \begin{itemize}
      \item No reflexiva pues $(a,a) \notin R$.
      \item No simétrica pues $(h,g) \in R$ pero $(g,h) \notin R$.
      \item No transitiva pues $e R c$ y $c R h$ pero $(e,h) \notin R$.
      \item No antisimétrica pues $a R b$ y $b R a$.
    \end{itemize}
  \item
    \begin{itemize}
      \item Reflexiva.
      \item No simétrica pues $(h,g) \in R$ pero $(g,h) \notin R$.
      \item No transitiva pues $c R h$ y $h R g$ pero $(c,g) \notin R$.
      \item No antisimétrica pues $a R b$ y $b R a$.
    \end{itemize}
  \item
    \begin{itemize}
      \item No reflexiva pues $(d,d) \notin R$.
      \item No simétrica pues $(h,g) \in R$ pero $(g,h) \notin R$.
      \item Transitiva.
      \item No antisimétrica pues $a R b$ y $b R a$.
    \end{itemize}
  \item
    \begin{itemize}
      \item Reflexiva.
      \item Simétrica.
      \item Transitiva.
      \item No antisimétrica pues $a R b$ y $b R a$.
    \end{itemize}
  \end{enumerate}
\item % Ejercicio 21
  Pendiente de completar.
\item % Ejercicio 22
  Pendiente de completar.
\item % Ejercicio 23
  Decimos que una relación es
  \begin{itemize}
  \item una relación de equivalencia sii es reflexiva, simétrica y transitiva.
  \item una relación de orden sii es reflexiva, antisimétrica y transitiva.
  \end{itemize}
  \begin{enumerate}
  \item Reflexiva, simétrica, transitiva, antisimétrica. Relación de orden y de equivalencia.
  \item No reflexiva, simétrica, transitiva, antisimétrica.
  \item Reflexiva, no simétrica, transitiva, antisimétrica. Relación de orden.
  \item Reflexiva, simétrica, transitiva, no antisimétrica. Relación de equivalencia.
  \item Reflexiva, no simétrica, transitiva, antisimétrica. Relación de orden.
  \item Reflexiva, no simétrica, transitiva, antisimétrica. Relación de orden.
  \item Reflexiva, no simétrica, transitiva y antisimétrica porque está definida con el operador $\subseteq$. Relación de orden.
  \end{enumerate}
\item % Ejercicio 24
  \begin{enumerate}
  \item Dado un conjunto $A$ y una relación $R$ en $A$
    \begin{itemize}
      \item Si $R = \emptyset$ entonces es simétrica y antisimétrica.
      \item Si $R$ es la relación de igualdad (o sea, la relación en $A$ solo reflexiva), también es simétrica y antisimétrica.
      \item Si a la relación de igualdad le quitamos algunos elementos, sigue siendo simétrica y antisimétrica.
    \end{itemize}
    Entonces, una relación $R$ en $A$ es simétrica y antisimétrica sii $R \subseteq \{(a,a) : a \in A\}$.
  \item Para ser una relación de equivalencia y de orden, además de ser simétrica y antisimétrica debe ser reflexiva y transitiva. La relación de igualdad es la única que es simétrica, antisimétrica y reflexiva. Además, también es transitiva. Luego, la única relación de equivalencia y de orden es la relación de igualdad.
  \end{enumerate}
  Una relación puede no ser simétrica ni antisimétrica. Por ejemplo, la relación del ejercicio 20. i).
\item % Ejercicio 25
  Sea $R$ una relación de equivalencia en el conjunto $A$ y sea $a \in A$. La clase de equivalencia de $a$ es el conjunto de elementos de $A$ que se relacionan con él y la notamos $\overline{a}$. Es decir, $\overline{a} = \{b : b \in A, b R a\}$\newline
  La partición asociada a $R$ es el conjunto de clases de equivalencia. Notar que dos elementos $x, y \in A$ pueden tener la misma clase de equivalencia, o sea $\overline{x} = \overline{y}$. En ese caso solo escribimos uno de los dos al describir la partición por extensión para que sea lo más corta posible.\newline
  No voy a escribir las clases de equivalencia y la partición de este ejercicio porque es muy tedioso.
\item % Ejercicio 26
  El primer ítem nos dice que podemos separar los números naturales según su último dígito. Es decir, la partición de $R$ es $\{\overline{0}, \overline{1}, \overline{2}\ldots\ \overline{9}\}$.\newline
  Pero como $(1,2) \in R$ y es una relación de equivalencia, entonces $\overline{1} = \overline{2}$. Siguiendo el mismo razonamiento, 
\end{enumerate}